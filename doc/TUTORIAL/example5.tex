\section{BALL Strings}

BALL provides a heavy-weight string class that has been designed
to provide a wealth of functionality using a simple and consistent syntax.
In general, you should avoid using {\tt const char*} or STL string when using
BALL, although they are compatible to each other. You can easily convert BALL
strings to char pointers (using the {\tt c\_str()} method and automatically
convert char pointers to BALL strings.

This part of the tutorial will give a short introduction to the wealthy
functionality of BALL strings. For complete information refer to the BALL
Referenced Manual.

\subsection{String Operations}

There are useful operations possible with BALL strings. Let us start with a
very basic one, concatenation. The following code snippet will concatenate two
BALL strings:
\begin{lstlisting}{}
String A("Concat");
String B("enate");
String C = A + B;
\end{lstlisting}
But concatenation is also defined with STL strings and even standard C
strings, \ie {\tt char*}, as operands:
\begin{lstlisting}{}
string A("Concat");
char* B = "enate";
String C = A + B;
\end{lstlisting}

Another very useful operation is swapping two strings:
\begin{lstlisting}{}
String A("Swap");
String B("swaP");
A.swap(B);
\end{lstlisting}

Something we might also use very often is reversing a string:
\begin{lstlisting}{}
String A("Swap");
A.reverse();
\end{lstlisting}

And finally, it is even possible to substitute parts of a string with another
String by using the \method{substitute} method:
\begin{lstlisting}{}
String A("Please replace REPLACE with something else.");
String B("REPLACE");
String C("SOMETHING ELSE");
A.substitute(B, C);
\end{lstlisting}


\subsection{Conversion}

BALL strings are featured with many conversion mechanisms. They are either
implemented through constructors or by one of the many explicit conversion
methods available. We will first have a look at some constructors taking other
types as argument. Let us first construct a String from some basic C types:
{\tt char}:
\begin{lstlisting}{}
char c_char = 'B';
int c_int = 1;
float c_float = 2.99792458;

String A(c_char);
String B(c_int);
String C(c_float);
\end{lstlisting}
There are many other simple types supported, like {\tt unsigned int}, {\tt
double}, etc. Refer to the reference manual for further information.

Let us turn to explicit conversions. 

\subsection{Predicates}

BALL Strings provide many predicates that can be used for determining special
properties. One ca find out, whether a String contains a certain substring, 
starts with a special prefix, ends with a suffix, consists only of letters or 
is simply a floating point number. The following code snippet will give you
some idea of the power of the predicates.

\subsection{Comparing Strings}

Commonly one often wants to compare strings, which was a pain with C type
character fields. BALL strings provide you with a simple interface and rih
functionality. 
